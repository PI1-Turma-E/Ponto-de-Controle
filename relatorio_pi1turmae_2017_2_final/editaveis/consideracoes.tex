\chapter{Conclusão}
O projeto designado a turma foi de implementação de sistema de telemedicina para exames de ultrassom a distância. A solução encontrada pela equipe foi em um braço mecânico controlado a distância por um profissional. O controle se daria por uma unidade de medição inercial (chip com acelerômetro e giroscópio integrado), enquanto o braço mecânico, que seguraria o transdutor do aparelho de ultrassom, obedeceria os comandos que o profissional enviar pela unidade de medição inercial. Para esse fim, é necessário uma boa comunicação entre a unidade de medição inercial e o braço e também garantir a segurança do paciente, pensando nesse fato que os requisitos traçados acima foram definidos.

	Já para a construção do protótipo, não haverá um aparelho de ultrassom, pois é um aparelho razoavelmente caro que não bate com o orçamento da equipe, então para mostrar o funcionamento do braço foi pensado em por uma webcam. O braço será controlado por uma unidade de medição inercial (MPU-6050) ligado ao Arduino, será impresso em material 3D e terá aproximadamente 30 cm de altura. Para a movimentação do braço mecânico serão utilizados servomotores com capacidade suficiente, à definir, para movimentação do braço e ligados ao Arduino.
	
\nocite{inmetro2003}
\nocite{abesco1}
\nocite{aneel1}
\nocite{santos1}
\nocite{portalsol1}
\nocite{ultrasom1}
\nocite{bracomec}
\nocite{ultrasom2}
\nocite{mathiassen-frontiers2016}
\nocite{pierrot1999hippocrate}
\nocite{djangoframe}
\nocite{moreto2007controle}
\nocite{fanuc}
\nocite{maquinasele}