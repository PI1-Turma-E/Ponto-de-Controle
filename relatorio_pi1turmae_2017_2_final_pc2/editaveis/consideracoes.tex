\chapter{Conclusão}
O projeto proposto visa o desenvolvimento de um sistema de telemedicina para a realização a distância.Dessa forma, o projeto foi desenvolver um braço robótico capaz de segurar o transdutor do aparelho de ultrassonagrafia escolhido, o sonoscape A6, sendo que esse braço será controlado a distância por uma médico, que terá, em mãos, para realizar esse controle com uma carcaça que comportará o módulo de sensor inercial. Essa carcaça foi vista nas figuras \ref{des_fig2} e \ref{des_fig4}.

Assim, todo o desenvolvimento da parte estrutural, energética, eletrônica e de software foi feito com base nas restrições e considerando o Brasil como área de implementação e expostos durante o desenvolvimento.

Foram apresentados os desenhos da carcaça do sistema principal e os desenhos do braço do sistema escravo, assim como, foram também apresentados todos os componentes eletrônicos necessários para o funcionamento do sistema, bem como as razões das escolhas de cada material.

Além disso, as fontes energéticas foram analisadas e decididas para uma melhor utilização do sistema e visando a segurança do paciente, sistemas energéticos secundários foram analisados e impostos ao projeto com todas as suas especificações.Também, a forma de comunicação entre o sistema principal e o sistema escravo foi definida e códigos de implementação realizados, além de terem sido mostradas as telas que aparecerão em cada computador.

Esse projeto, por fim, é capaz de ajudar populações brasileiras que vivem em comunidades mais afastadas de grandes centros e que não possuem facilidade em conseguir realizar o exame, pois, com o projeto, esse exame poderá ser realizado a distância, facilitando assim a vida das pessoas que precisam desse exame, mas que, atualmente, não possuem acesso.
